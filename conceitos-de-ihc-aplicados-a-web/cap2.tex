\chapter{O que é User Experience (UX)}

\begin{flushright}
	\textit{
		UX é pesquisar sobre os usuários de modo que você possa dar para eles \\ o que eles precisam para conseguirem algo que querem.
	} \\
	
	\textbf{autor desconhecido}
\end{flushright}


No Capítulo \ref{cap:cap1}, nós realizamos a introdução sobre alguns conceitos norteadores da Interação Humano-Computador (IHC). No meio de todos os conceitos apresentados, algo se destaca em meio ao material apresentado, que é \textbf{usuário}. Sem dúvida, podemos afirmar que, o usuário, mediante sua interação com os sistemas computacionais, é a base para que uma área de estudo como IHC pudesse ser construída.

Já neste capítulo, iremos tratar não somente da interação do usuário por meios das interfaces, mas também da qualidade da sua experiência ao usar elas.

\section{Experiência do usuário}

Segundo \citeonline{teixeira2014introduccao}, a experiência do usuário existe desde que o mundo é mundo. Desde que as pessoas começaram a ``usar' objetos para realizar alguma tarefa podemos dizer que existe um contexto de experiências. Experiências são subjetivas. Cada pessoa tem uma experiência
diferente ao usar um caixa eletrônico, um aplicativo, uma rede social, entre outros. Essa experiência, segundo \citeonline{teixeira2014introduccao} é
influenciada por dois tipos de fatores: \textbf{humanos} e \textbf{externos} 

Como fatores humanos podemo citar as habilidade em usar caixas eletrônicos, por exemplo. Sua visão, sua habilidade motora, sua capacidade de ler e entender o que está
escrito na tela, seu humor naquele momento, entre outros fatos.Já como fatores externos, podemos citar o horário do dia, o ambiente onde o caixa eletrônico está instalado, o fato de ter uma fila de pessoas atrás do utilizador, e assim por diante.

Assim, podemos dizer que o tema experiência do usuário é um tema bastante subjetivo. É difícil de maneira objetiva e direta dizer como desenhar uma experiência do usuário, mas é possível aprendermos como desenhar um produto, serviço ou ambiente que proporcione uma experiência satisfatória para alguém que os use, identificando todos os aspectos da interação do usuário com esse produto (ou serviço ou ambiente).

Portanto, a experiência do usuário esta ligado a forma com que a pessoa se sente ao usar um produto. Ou mais formalmente, de acordo com a definição dada pela ISO 9241-210, são as respostas e percepções de uma pessoa
resultantes do uso de um produto, sistema ou serviço.
