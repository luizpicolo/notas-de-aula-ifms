\chapter{Variáveis e Constantes}

Definir o que são variáveis ou constantes de forma geral é extremamente simples. Uma variável nada mais é do que um espaço que o sistemas operacional (SO) aloca na memória de seu computador para que seja armazenado algo. Ou seja, quando precisamos armazenar um determinado valor nosso SO cria um espaço e permite ao usuário determinar o que será adicionado. A medida que não será utilizado este espaço com um valor, ele será dissipado, fazendo com que seus disponibilizando-os para serem usados na construção de outros novos valores \cite{haverbeke2014eloquent}.

Mas como é criado este valores? Eu preciso entender de memoria, sistemas operacionais, entre outros para poder alocar este valores? A resposta é não, não é necessário a compreenção de tão baixo nível, basta que se compreenda como é criar variáveis ou constantes em um determinada linguagem, que, neste caso, é JavaScript.

\section{Tipos de dados}

Diferente das linguagens com tipagem forte, ou seja, que levam em consideração o tipo para executar suas operações, o JavaScript possui tipagem fraca, a qual é dado o tipo dos dados que serão alocados quando o código é executado. Outro ponto é a tipagem dinâmica. Logo, não é necessário tipar uma variável quando inicia-se o programa como é feito em Java, C, C++, entre outras linguagens, mas esta responsabilidade é repassada ao compilador, que, caso do JavaScript em questão, esse processo é dado o nome Inferência de Tipo.

A inferência de tipos é a capacidade do compilador entender/’adivinhar’ qual é o tipo de dados de determinada variável sem ela ter sido declarada no código escrito. 

Segundo a \citeonline{mdnmozilla2019}, JavaScript reconhece os seguintes tipos de valores:

\begin{itemize}
  \item Números (en), como 42 ou 3,14159
  \item Valores lógicos (Booleanos) (en), true ou false
  \item Strings (en), tais como "Howdy!"
  \item null, um palavra chave especial denotando um valor nulo; null também é um valor primitivo. Como JavaScript é sensível a maiúsculas, null não é a mesma coisa que Null, NULL, ou qualquer outra variante
  \item undefined (en), uma propriedade de alto nível a qual possui o valor indefinido; undefined também é um valor primitivo.  
\end{itemize}

Este conjunto de tipos de valores, relativamente pequeno, permite realizar funções úteis nas aplicações. Não há distinção explícita entre números inteiros e reais, para JavaScript, todos os números são tratados como \textit{Number}. 

\section{Variáveis}

O JavaScript é uma linguagem de tipagem dinâmica. Isto significa que não é necessário especificar o tipo de dado de uma variável quando ela for declarada, e tipos de dados são automaticamento convertidos conforme necessário durante a execução do script. Então, por exemplo, pode-se definir uma variável como:

\begin{lstlisting}
  var idade = 30
  var instituicao = "IFMS" 
\end{lstlisting}

Para que possamos saber o tipo que foi atribuido para cada variável, podemos utilizar a função \textbf{typeof} da seguinte forma: 

\begin{lstlisting}
  var idade = 30
  console.log(typeof idade) // Retornará number
\end{lstlisting}

Outra forma de declarar variáveis é utilizando a palavra reservada \textbf{let}.

\begin{lstlisting}
  let idade = 30
  let instituicao = "IFMS" 
\end{lstlisting} 

Mas qual a diferença entre \textbf{var} e \textbf{let}? Neste momento não possuimos o conhecimento necessário para diferenciar as duas palavras reservadas. Contudo, veremos nos próximos capítulos como as duas se diferenciam e quando utilizar uma ou outra.

\subsection{Hoinsting}

Segundo a \citeonline{mdnmozilla2019}, em JavaScript, funções (a qual será vista em um outro capítulo) e variáveis são hoisted (ou "levados ao topo"). Hoisting é um comportamento do JavaScript de mover declarações para o topo de um escopo (o escopo global ou da função em que se encontra). Isso significa que nós somos capazes de usar uma função ou variável antes mesmo de tê-las declaradas, ou em outras palavras: uma função ou variável podem ser declaradas depois de já terem sido utilizadas.

\begin{lstlisting}
  foo = 2
  var foo;

  // é implicitamente entendido como:
  var foo;
  foo = 2;
\end{lstlisting}

Assim, as variáveis quando são declaradas sem um valor, recebem o valor \textbf{undefined}, ou seja, sem tipo.

\section{Constantes}

Podemos criar elementos "somente leitura", nomeados constantes com a palavra chave \textbf{const}. A sintaxe de um identificador constante é a mesma para um identificador de variáveis: deve começar com uma letra ou sublinhado e pode conter caracteres alfabéticos, numéricos ou sublinhado. Porém, uma constante não pode ter seu valor mudado por meio de uma atribuição ou ser declarada novamente enquanto o \textit{script} estiver rodando.

\begin{lstlisting}
  const idade = 30
  console.log(typeof idade) // Retornará number
\end{lstlisting}