\chapter{Mapeamento de objetos para o Modelo Relacional}\label{cap:cap1}

\begin{flushright}
	\textit{
		Um ladrão rouba um tesouro, mas não furta a inteligência. \\
		Uma crise destrói um herança, mas não uma profissão. \\ Não importa se você não tem dinheiro, você é uma pessoa rica, \\ pois possui o maior de todos os capitais: a sua inteligência. \\ Invista nela. \textbf{Estude}!.
	} \\
	
	\textbf{Augusto Cury}
\end{flushright}

A tecnologia de orientação a objetos é consolidada como a forma mais usual para o desenvolvimento de softwares. Já, quando pensamos em banco de dados, os bancos relacionais forma os que tiveram maior exito e, sem dúvida, os Sistemas Gerenciadores de Banco de dados Relacionais (SGBDR) dominam o mercado. 

No entanto, essas duas tecnologias surgiram por meio de princípios teóricos muito diferentes. 

Segundo \citeonline{ferreira2009banco}, MER (Modelo entidade-relacional) é um modelo de dados conceitual de alto-nível, ou seja, seus conceitos foram projetados para serem compreensíveis a usuários, descartando detalhes de como os dados são armazenados. Atualmente, o MER é usado principalmente durante o processo de projeto da base de dados.

\section{Conceitos do Modelo entidade-relacionamento}

O objeto básico que o MER representa é a \textbf{entidade}. Uma entidade é algo do mundo real que possui uma existência independente. Uma entidade pode ser um objeto com uma existência
física - uma pessoa, carro ou empregado - ou pode ser um objeto com existência conceitual -
uma companhia, um trabalho ou um curso universitário. Cada entidade tem propriedades
particulares, chamadas \textbf{atributos}, que a descrevem \cite{ferreira2009banco}. 
