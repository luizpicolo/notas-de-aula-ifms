\chapter{Versionamento de Código}\label{cap:cap3}

\begin{flushright}
	\textit{
		Um ladrão rouba um tesouro, mas não furta a inteligência. \\
		Uma crise destrói um herança, mas não uma profissão. \\ Não importa se você não tem dinheiro, você é uma pessoa rica, \\ pois possui o maior de todos os capitais: a sua inteligência. \\ Invista nela. \textbf{Estude}!.
	} \\
	
	\textbf{Augusto Cury}
\end{flushright}

De forma simples, podemos dizer que o controle de versão é um sistema que registra as mudanças feitas em um arquivo ou um conjunto de arquivos ao longo do tempo de forma que você possa recuperar versões específicas para praticamente qualquer tipo de arquivo em um computador. 

Assim, esse capítulo, tem como objetivo proporcionar a você leitor(a) os pontos os quais julgamos mais importantes para compreender e praticar o controle de versão. Esse tipo de técnica é amplamente utilizado em ambientes de trabalho, nos quais, várias pessoas tem que trabalhar em um mesmo projeto. Assim, caso um erro aconteça, ou características nova surjam, os membros da equipe podem corrigir o problema de forma simples e rápida.  

\section{Projeto de banco de dados}

