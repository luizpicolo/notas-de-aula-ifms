\chapter{Definição sobre JavaScript Object Notation (JSON)}\label{cap:cap3}

\begin{flushright}
	\textit{
		Um ladrão rouba um tesouro, mas não furta a inteligência. \\
		Uma crise destrói um herança, mas não uma profissão. \\ Não importa se você não tem dinheiro, você é uma pessoa rica, \\ pois possui o maior de todos os capitais: a sua inteligência. \\ Invista nela. \textbf{Estude}!.
	} \\
	
	\textbf{Augusto Cury}
\end{flushright}

JSON (JavaScript Object Notation) é um formato de intercâmbio de dados leve. É fácil para que seres humanos possam ler e escrever. É fácil para que as máquinas possam analisar e gerar. É baseado em um subconjunto do \textbf{Padrão de Linguagem de Programação JavaScript ECMA-262 3ª Edição} - Dezembro de 1999. JSON é um formato de texto completamente independente da linguagem, mas usa convenções que são familiares aos programadores da família C de linguagens, incluindo C, C ++, Java, JavaScript, Perl, Python e muitos outros. Essas propriedades tornam o JSON uma linguagem de intercâmbio de dados ideal\footnote{Para mais detalhes acesse: https://www.json.org/json-pt.html}.

Json também é um formato baseado em texto padrão para representar dados estruturados com base na sintaxe do objeto JavaScript. É comumente usado para transmitir dados em aplicativos da Web (por exemplo, enviar alguns dados do servidor para o cliente, para que possam ser exibidos em uma página da Web ou vice-versa). Você se deparará com isso com bastante frequência, portanto, neste artigo, oferecemos tudo o que você precisa para trabalhar com o JSON usando JavaScript, incluindo a análise do JSON para que você possa acessar os dados dentro dele e criar o JSON\footnote{Para mais detalhes acesse: \url{https://developer.mozilla.org/pt-BR/docs/Learn/JavaScript/Objects/JSON}}.

\section{Estrutura JSON}

Um JSON é uma string cujo formato se parece muito com o formato literal do objeto JavaScript. Você pode incluir os mesmos tipos de dados básicos dentro do JSON, como em um objeto JavaScript padrão — strings, números, matrizes, booleanos e outros literais de objeto. Isso permite que você construa uma hierarquia de dados. 

\begin{lstlisting}
let Retangulo = {
	"largura": 100,
	"altura": 100
}

let Retangulos = [
	{"largura": 100, "altura": 100},
	{"largura": 200, "altura": 200},
]
\end{lstlisting}

Assim, para realizar o acesso aos dados de um objeto Json, usamos a mesma notação \textbf{dot} / \textbf{bracket} usada em objetos literais.

\begin{lstlisting}
let Retangulo = {
	"largura": 100,
	"altura": 100
}

let Retangulos = [
	{"largura": 100, "altura": 100},
	{"largura": 200, "altura": 200},
]

console.log(Retangulo.largura)
console.log(Retangulo['largura'])

console.log(Retangulos[0].largura)
console.log(Retangulo[0]['largura'])
\end{lstlisting}

Contudo, o Json recebido venha em formato de String, ele deverá ser convertido para um objeto JavaScript. Assim, usando a função \textbf{JSON.parse()} podemos converter a string em um objeto, da seguinte forma:

\begin{lstlisting}
let retangulo = Json.parse('{"largura": 100, "altura": 100}')
console.log(Retangulo.largura)
\end{lstlisting}

\section{Exercícios de Fixação}

\begin{enumerate}
	\item No exercício 2.4 criamos uma classe \textbf{Noticia} que continha alguns atributos (titulo, dataDaPublicacao, resumo e texto) e uma outra classe \textbf{NoticiaDestaque} que continha, além dos atributos citados, um atributo imagem. Assim, baseando-se nos atributos listados, crie um arquivo \textbf{json} com nome \textbf{noticias.json} e atribua a ele uma lista de notícias no formato estudado. O arquivo deve conter no mínimo 5 notícias.
\end{enumerate}

\section{Obtendo JSON de um servidor web (Requisição Ajax)}

Para obter o JSON, vamos usar uma API chamada \textbf{XMLHttpRequest} (geralmente chamada de XHR). Esse é um objeto JavaScript muito útil que nos permite fazer solicitações de rede para recuperar recursos de um servidor via JavaScript (por exemplo, imagens, texto, JSON e até trechos de código HTML), o que significa que podemos atualizar pequenas seções de conteúdo sem ter que recarregar todo página. Isso levou a páginas da Web mais responsivas e parece empolgante, mas está além do escopo deste artigo ensinar isso com muito mais detalhes\footnote{Para mais detalhes acesse: \url{https://developer.mozilla.org/pt-BR/docs/Learn/JavaScript/Objects/JSON}}.

Para começar, vamos armazenar a URL do JSON que queremos recuperar em uma variável. Adicione o seguinte na parte inferior do seu código JavaScript:

\begin{lstlisting}
	let requestURL = 'https://raw.githubusercontent.com/luizpicolo/json-for-testing
/main/dragonball.json';
\end{lstlisting}

Para criar uma solicitação, precisamos criar uma nova instância de objeto de solicitação a partir do construtor XMLHttpRequest usando a palavra-chave new. Adicione o seguinte abaixo sua última linha:

\begin{lstlisting}
	let request = new XMLHttpRequest();
\end{lstlisting}

Agora precisamos abrir uma nova solicitação usando o método open() . Adicione a seguinte linha:

\begin{lstlisting}
	request.open('GET', requestURL);
\end{lstlisting}

Isso leva pelo menos dois parâmetros — existem outros parâmetros opcionais disponíveis. Nós só precisamos dos dois obrigatórios para este exemplo simples:

\begin{itemize}
	\item O método HTTP a ser usado ao fazer a solicitação de rede. Neste caso, GET é bom, pois estamos apenas recuperando alguns dados simples.
	\item O URL para fazer a solicitação — esta é a URL do arquivo JSON que armazenamos anteriormente.
\end{itemize}

Em seguida, adicione as duas linhas a seguir — aqui estamos definindo o  responseType como JSON, para que o XHR saiba que o servidor retornará o JSON e que isso deve ser convertido nos bastidores em um objeto JavaScript. Em seguida, enviamos a solicitação com o método send():

\begin{lstlisting}
	request.responseType = 'json';
	request.send();
\end{lstlisting}

A última parte desta seção envolve aguardar a resposta retornar do servidor e, em seguida, lidar com ela. Adicione o seguinte código abaixo do seu código anterior:

\begin{lstlisting}
	request.onload = function() {
		let personagens = request.response;
		
		const elemento = document.getElementById('list');
		let p1 = personagens.characters[0].name;
		elemento.insertAdjacentHTML('afterbegin', p1);
	}
\end{lstlisting}

\section{Exercícios de Fixação}

\begin{itemize}
	\item A partir do arquivo \textbf{noticias.json} criado no exercício anterior, e utilizando também as classes \textbf{Noticia} e \textbf{NoticiaDestaque} implementadas, leia e apresente no navegador as notícias contidas no arquivo \textbf{noticias.json}. Para tanto, você deve utilizar implementar a leitura do arquivo e, a partir dos dados, criar objetos e apresentar os mesmos no navegador. Abaixo segue o modelo base para o desenvolvimento do trabalho
\end{itemize}