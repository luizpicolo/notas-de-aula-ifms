\chapter{Mapeamento objeto-relacional (ORM)}\label{cap:cap3}

\begin{flushright}
	\textit{
		É melhor você tentar algo, vê-lo não funcionar \\ e aprender com isso, do que não fazer nada.
	} \\
	
	\textbf{Mark Zuckerberg}
\end{flushright}

Criar um banco de dados, ou mais formalmente o Sistema Gerenciador de Banco de Dados (SGBD), é uma tarefa complexa presente em todos os projetos. Escolher qual será utilizar em meio a tantos como: my\textbf{sql, sqlserver, oracle, sqlite , postgre, mongodb e etc}, ou, qual se adapta melhor ao projeto é uma decisão que deve ser tomada com cautela.

Outro fato é a forma com que os dados são tratados em cada um. Por exemplo, para um campo \textbf{id} que deve ser auto-incrementado, no Postgres usando o seguinte código SQL:

\begin{minted}[frame=single,framesep=10pt,breaklines,linenos,tabsize=2,autogobble]{sql}
	CREATE TABLE Pessoas (
		id SERIAL NOT NULL,
		nome varchar
	);
\end{minted}

Já no Mysql, o mesmo código seria escrito da seguinte forma:

\begin{minted}[frame=single,framesep=10pt,breaklines,linenos,tabsize=2,autogobble]{javascript}
	CREATE TABLE Pessoas (
		id int NOT NULL AUTO_INCREMENT,
		nome varchar
	);
\end{minted}

Logo, caso houvesse uma mudança, nosso código teria que ser alterado para satisfazer a nova base de dados com todas as suas diferenças.

Assim, pensando nas possíveis mudanças que o projeto pode ter durante sua vida útil, foram desenvolvidas algumas técnicas para facilitar a vida dos desenvolvedores. Uma que iremos comentar é o Object Relational Mapping (ORM) ou Mapeamento Objeto Relacional. Para diminuir a complexidade, já que o ORM torna o banco de dados mais próximo da arquitetura de classe, removendo os comando SQL de vista, para que possamos focar em ``Qual é o fluxo que minha aplicação deve seguir'' e deixando de lado ``Qual é a query que eu deveria usar aqui?''. Então, nesse aspecto, iremos abordar um pouco sobre o que são ORM, suas práticas e exemplos usando JavaScript.

\section{ORM: o que é e como funciona? }

Como já citei anteriormente um ORM é um Mapeamento Objeto Relacional, sua base consiste em manter o uso de orientação a objetos e um pouco do conceito de non-query. Pois serão raros os momentos onde teremos que escrever uma linha de código SQL para esse tipo de ferramenta \cite{Aylon2020Muramatsu}.

Outro fato muito importante e curioso sobre os ORM é que eles operam como um agente de banco de dados, sendo possível através de pouquíssimas mudanças, utilizar o mesmo código para mais de um banco de dados. Não importa se ele está em Mysql, SqlServer ou até mesmo Oracle. Ele consegue agir da mesma forma em alguns bancos de dados, você só precisa mudar o driver de conexão e está pronto para uso. Neste conceito é importante lembrar que cada uma de nossas tabelas são vistas como uma instância de uma classe, tendo suas características declaradas diretamente na sua classe ``esquema''. Quando trabalhamos com esse tipo de esquema sempre teremos um arquivo de configuração, responsável por fornecer os dados para que o componente de ORM possa se comunicar com o banco e aplicação. Uma outra questão que pode causar dúvidas é como gerar o banco de dados através dessas classes que comentamos anteriormente. Bom veremos isso na prática mais a abaixo \cite{Aylon2020Muramatsu}.